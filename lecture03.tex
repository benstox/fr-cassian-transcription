\chapter{\textit{Nihil Operi Dei praeponatur}}

\begin{itshape}
In the name of the Father and of the Son and of the Holy Spirit. Amen.

\smallskip

\noindent Hail Mary, full of grace, the Lord is with thee. Blessed art thou among women and blessed is the fruit of thy womb, Jesus.

\smallskip

\noindent Holy Mary, Mother of God, pray for us sinners, now and at the hour of our death. Amen.

\smallskip

\noindent Saint Benedict, pray for us.

\smallskip

\noindent In the name of the Father and of the Son and of the Holy Spirit. Amen.

\end{itshape}

\smallskip

We continue with our look at the Divine Office, and in particular the psalmody, asking the question, how does the Divine Office fit into this notion of prayer without ceasing. And more particularly then, how do the psalms fit into this notion of prayer without ceasing. We can also ask the question, how do the psalms fit into this Christocentric notion of prayer that we've been talking about, that is, Christian prayer as entering into the dialogue existing outside of us between the Father and the Son. So those are the basic questions and I'd like to propose three basic answers to those questions: one from St Benedict, one from St Athanasius, and one from St Augustine. The handout you have deals with two of those responses. There's no handout for St Athanasius.

So let's begin with St Benedict. What is the \emph{opus Dei} about, the Divine Office? Let's look at the first citation there from chapter 43 where in a very famous phrase he says, ``Let nothing be prefered to the work of God''. Now as we know there are two other passages in the Rule that sound very much the same. In chapter 4 he says, ``to prefer nothing to the love of Christ''. And then in chapter 72, ``let the prefer absolutely nothing to Christ''. Now when we're doing textual criticism, frequently enough you line up texts that are very similar and you look for where they're different, because where they're different is going to give you some clue as to what the meaning is. And in these three texts we have exactly the same verb, \emph{praeponere}, and the exact same subject, nothing, but the object changes in these three instances. Let nothing be prefered to what? Well, to the work of God, to the love of Christ, and to Christ. It's a valid conclusion to draw that these three terms \textemdash\ the objects, to prefer nothing to such-and-such \textemdash\ that those three terms are somehow interchangeable, or at least shed light upon one another. For our purposes, I'd like to propose that they're practically synonyms. That is, the work of God is the same thing as the love of Christ, is the same thing as Christ himself. So the reasoning goes this way. The work of God to which we are to prefer nothing has something to do with Christ. The psalms are the core of the work of God. The psalms to which we are to prefer nothing have something to do with Christ. So we come to a fundamental patristic principle of scriptural interpretation




