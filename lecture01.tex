\documentclass[10pt,a5paper]{book}

\begin{document}

O lux beata Trinitas

Fr Cassian was invited to St Benedict's Abbey in Still River, Massachussets in the early October 2013 to give a series of conferences on praying without ceasing. Drawing on Scripture, the Liturgy and the Fathers, Fr Cassian gives us some tips on how to pray better. In this day and age, we want quick fixes and easy answers, but we can only learn to pray better by doing it more frequently and this takes time.

Fr Cassian is the founder and prior of the Monastery of San Benedetto, located at the birthplace of St Benedict in Norcia, Italy. Fr Cassian has remained on faculty at the Pontifical Liturgical Institute in Rome, and since 2010 has been a consultor to the Congregation of Divine Worship and the Discipline of the Sacraments. Fr Cassian became a Benedictine monk at St Meinrad in southern Indiana in 1980 and was ordained a priest in 1984. He earned his doctorate in Sacred Liturgy from Sant Anselmo in 1989. He has given retreats and talks throughout his life as a monk.

\chapter{Praying without Ceasing}

In the name of the Father and of the Son and of the Holy Spirit.

Let us pray.

Pour forth, we beseech Thee, O Lord, Thy Grace into our hearts, that we, to whom the Incarnation of Christ, Thy Son, was made known by the message of an angel, may through His Passion and Cross be brought to the glory of His Resurrection. Through the same Christ our Lord.

Well, I'm very happy to be with you again. As we were talking about last night at recreation, there's a long history of connection and friendship which I value very much. It seems strange perhaps that monks need a retreat, but in fact our lives are very busy and rather intense, so a retreat is a wonderful thing in which you can respond to the Lord's invitation to, ``come away to a quiet place and rest a while''. So I'm sure getting some extra rest is important this week, but also a time to reflect, a kind of review of the past year. And we tend to ask ourselves questions like, where am I in the spiritual life? And what is God doing in my life? What are my preoccupations? What changes do I need to make? Maybe I need to reorient myself, to get back on track. How is my life of prayer? The question that you need to ask God will give you to ask, but we need to ask them.

The topic of the retreat is prayer. It's a general theme suggested by Abbott Xavier in an e-mail to me. He was talking about the relationship between Mass and Office in particular, if I remember correctly. And I was thinking about that, and thought, well, that's crucial but it has to fit into a larger picture, because there's also Lectio, there's private prayer, there's all kinds of other things. We're searching for unity in our spiritual life, not fragmentation. So what I'd like to do, or try to do, is provide that unifying context in which we can examine more particularly the various aspects of our life of prayer. That's why I gave you this little outline here, so you'll know where we're going.

The question of prayer is basic to our life and it's a response to Our Lord. We call the disciples' request of Jesus, ``Lord, teach us how to pray''.  It's very curious the way he goes about responding to that question. And there's a little icon, which describes the scene. A very beautiful one. I'll just pass it around --- show-and-tell --- so you have something to do. But this icon is also in the third part of the Catechism in the section on prayer. That is, each of those four sections has some kind of image from the tradition which summarizes the whole content of that section. And this is the very image that's in the Cathechism in the section on prayer.

The disciples ask Jesus to teach them to pray, so you'd think that he would turn toward them and give them some sort of instruction, and engage them in some sort of dialogue. But he doesn't. In the icon he's not facing the disciples at all. He turns and faces the Father. The Father is pictures up in a little cloud, because a cloud is the symbol of divinity. And he addresses himself to the Father. And the disciples observe that. In the icon you can see.    And it's a perfect definition of prayer. That is, not something that we do, but somehting we participate in. Prayer is the dialogue between the Father and the Son, and the communion of the Holy Trinity. Whether you pray or not, the Son is praying to the Father.

Gospel injunction to pray without ceasing. Luke 18 parable

\end{document}
