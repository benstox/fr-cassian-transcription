\documentclass[10pt,a5paper]{book}

\begin{document}

O lux beata Trinitas

Fr Cassian was invited to St Benedict's Abbey in Still River, Massachussets in the early October 2013 to give a series of conferences on praying without ceasing. Drawing on Scripture, the Liturgy and the Fathers, Fr Cassian gives us some tips on how to pray better. In this day and age, we want quick fixes and easy answers, but we can only learn to pray better by doing it more frequently and this takes time.

Fr Cassian is the founder and prior of the Monastery of San Benedetto, located at the birthplace of St Benedict in Norcia, Italy. Fr Cassian has remained on faculty at the Pontifical Liturgical Institute in Rome, and since 2010 has been a consultor to the Congregation of Divine Worship and the Discipline of the Sacraments. Fr Cassian became a Benedictine monk at St Meinrad in southern Indiana in 1980 and was ordained a priest in 1984. He earned his doctorate in Sacred Liturgy from Sant Anselmo in 1989. He has given retreats and talks throughout his life as a monk.

\chapter{Praying without Ceasing}

In the name of the Father and of the Son and of the Holy Spirit.

Let us pray.

Pour forth, we beseech Thee, O Lord, Thy grace into our hearts, that we, to whom the Incarnation of Christ, Thy Son, was made known by the message of an angel

Well, I'm very happy to be with you again. As we were talking about last night at recreation, there's a long history of connection and friendship which I value very much. It seems strange tha monks need a retreat, but in fact our lives are rather busy and rather intense, so a retreat is a very wonderful thing in which you can respond to the Lord's invitation to, ``come away to a quiet place and rest a while''. So I'm sure getting some extra rest is important this week, but also some time to reflect on the past year

...

The questions that we need to ask ...


The topic of the retreat is prayer. It's a general theme suggested by Abbott Xavier in an e-mail

Gospel injunction to pray without ceasing. Luke 18 parable

\end{document}
