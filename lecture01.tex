\chapter{Praying without ceasing}

\pagestyle{fancy}
\fancyhf{}
\fancyhead[LE]{\itshape\nouppercase{\thetitle\ --- \theauthor}}
\fancyhead[RO]{\itshape\nouppercase{\leftmark}}
\fancyfoot[CE,CO]{\thepage}

In the name of the Father and of the Son and of the Holy Spirit.

Let us pray.

Pour forth, we beseech Thee, O Lord, Thy Grace into our hearts, that we, to whom the Incarnation of Christ, Thy Son, was made known by the message of an angel, may through His Passion and Cross be brought to the glory of His Resurrection. Through the same Christ our Lord.

Well, I'm very happy to be with you again. As we were talking about last night at recreation, there's a long history of connection and friendship which I value very much. It seems strange perhaps that monks need a retreat, but in fact our lives are very busy and rather intense, so a retreat is a wonderful thing in which you can respond to the Lord's invitation to \emph{come away to a quiet place and rest a while}. So I'm sure getting some extra rest is important this week, but also a time to reflect, a kind of review of the past year. And we tend to ask ourselves questions like, where am I in the spiritual life? And what is God doing in my life? What are my preoccupations? What changes do I need to make? Maybe I need to reorient myself, to get back on track. How is my life of prayer? The question that you need to ask God will give you to ask, but we need to ask them.

The topic of the retreat is prayer. It's a general theme suggested by Abbott Xavier in an e-mail to me. He was talking about the relationship between Mass and Office in particular, if I remember correctly. And I was thinking about that, and thought, well, that's crucial but it has to fit into a larger picture, because there's also Lectio, there's private prayer, there's all kinds of other things. We're searching for unity in our spiritual life, not fragmentation. So what I'd like to do, or try to do, is to provide that unifying context in which we can examine more particularly the various aspects of our life of prayer. That's why I gave you this little outline here, so that you'll know where we're going.

The question of prayer, of course, is basic to our life and it's a response to Our Lord. We call the disciples' request of Jesus, ``Lord, teach us how to pray''.  It's very curious the way he goes about responding to that question. And there's a little icon, which describes the scene. A very beautiful one. I'll just pass it around --- show-and-tell --- so you have something to do. But this icon is also in the third part of the Catechism in the section on prayer. That is, each of those four sections has some kind of image from the tradition which summarizes the whole content of that section. And this is the very image that's in the Cathechism in the section on prayer.

The disciples ask Jesus to teach them to pray, so you'd think that he would turn toward them and give them some sort of instruction, and engage them in some sort of dialogue. But he doesn't. In the icon he's not facing the disciples at all. He turns and faces the Father. The Father is pictured up in a little cloud, because the cloud is the symbol of divinity. And he addresses himself to the Father. And the disciples observe that. You can see in the icon they're watching to see what's going on there. And it's a perfect definition of prayer: that is, not something that we do, in the first place, but somehting that we participate in. Prayer is the dialogue between the Father and the Son, and the communion of the Holy Trinity. And we are invited in to that. It's going on quite independently of us. So Whether we pray or not, the Son is praying to the Father. And so, just as in the liturgy, it's not so much something that we do, but the heavenly liturgy, in which we participate. So also in other kinds of prayer, it's the key relationship between the Father and the Son to which we are invited.

So monastic prayer is nothing more than a deepening of this Christian prayer. The fact is that we dedicate time and energy and the orientation of our whole life to prayer. Cassian describes this preoccupation of the monk with the things of God and our desire to participate in the prayer between Jesus and the Father in these words: ``The goal is that we shall see the fulfillment of our Saviour's prayer to the Father for his disciples''. And he quotes John's Gospel, ``May the love with which you have loved me,'' Jesus says to the Father, ``be in them and they in us. May they be one in you, Father, as you are in me and I am in you''. That they also may be one in us. So this communion with the Trinity is the goal of our prayer. This perfect love with which God has first loved us will come about in our hearts through the fulfillment of this prayer of the Lord. And our faith tells us that his prayer cannot be in vain. So what's key is the prayer of `Christ. These are the signs that will accompany the fulfillment, the signs in us. God will be our whole love and desire. Our whole study and labour. Our whole thought, our whole life. Our speaking. Our breathing. The unity which exists between the Father and the Son will be communicated to our souls and emotions. Just as God loves us with a true and pure and unfailing charity, so we will be united to him with the indissoluble unity of constant love. We will be so attached to him that our whole yearning and thinking and speaking will be about him alone. So the life of prayer to which we aspire is something that embraces us in a total kind of way.

I'd like to being with  the desert tradition because the retreat is entitile Prayer in the Monastic Tradition and the Desert Tradition is where we get the beginnings of this kind of theology of preayer that will be developed in the later tradition. Now the Desert Tradition doesn't only mean Egypt, it also includes Palestine and Syria and Asia Minor. It's that whole area. But anyways, those first few centuries of monastic experience.

The Desert Fathers were really concerned with the Gospel injunction to pray without ceasing. In Luke 18, the parable about the pesty widow who molests the judge with her constant petitions, the beginning of that parable says, the Lord gave the parable so that we would learn to pray without ceasing and never give up. Also in 1 Thessalonians 5:17, St Paul lists a number of desirable qualities: \emph{rejoice always and pray without ceasing}. We know that the Desert Fathers took this seriously, and there are some wonderful stories that describe this, so I'll be telling a lot of stories in this first conference. They're more interesting anyway. First, about Arsenius: ``Having withdrawn to the solitary life, Arsenius made this prayer: `Lord, lead me in the way of salvation'. And he heard a voice saying to him, `Arsenius, flee, be silent, pray always, for these are the sources of sinlessness'''. So part of the vocation of Arsenius is to pray without ceasing. There's another Desert Father, Benjamin, who talks about the same thing. ``As he was dying, Abba Benjamin said to his sons, `If you observe the following you can be saved,''' and he quotes 1 Thessalonians 5, ```\emph{be joyful at all times, pray without ceasing, and give thanks for all things}'''.

This ideal of praying without ceasing is very attractive, but how can we achieve it? It's one thing to present the ideal, it's another thing to get there. We're familiar with the story of the Russian pilgrim, \emph{The Way of the Pilgrim}, which popularised the Jesus prayer, and you might recall that in the beginning of this pilgrim's search he was looking for a wise old man who could tell him how to pray without ceasing. And he heard that there was a certain hermit living on a country estate, and he went and found him and said, ```I beg you in God's name to explain to me the hidden meaning of the Apostle's words, \emph{pray without ceasing}. How is it possible to pray continually?' The man gazed at me for a long time before speaking.'' There's a certain reluctance to communicate these secrets. In Casssian's report in Conferences 9 and 10 where Abba Isaiah is the one speaking, Cassian and Germanus, his companion in crime, ask Abba Isaiah to teach them how to pray. And he says, well, I'm really not sure I want to tell you. I'm paraphrasing of course. But the idea is that you have to be serious about it. You can't throw you pearls before the swine. If somebody is not serious, I'm not going to tell you. There's no point in bothering about it. So, there's a kind of theme, a constant theme in the tradition, of reluctance to explain these intimate things about our relationship with God. For us that's rather strange. We think of evangelisation and the need to preach and give them their method and all that sort of thing. The Fathers are a bit more reserved. And there's a wisom to that too. If you're serious about this, then it'll cost you something.

''So the old man said to the pilgrim, 'Continual interior prayer is the spontaneous leaping forth of man's spirit toward God.''' Now he's not going to give him a method right away. He says, '''You have to pray to God to enlighten you about the means by which the Spirit may accomplish this activity. Pray hard and fervently. Prayer itself will reveal to you how it can be prolonged unceasingly. But this will take much time.''' So there's no magical recipe for prayer, and we learn about prayer by praying. Those are good principles.

Now in the description I'm going to give about the Desert Teaching, I'm going to use a systematic framework that was provided by a Patristic scholar named Iren\'{e} Orsaire, who has a little book, well, not so little, called \emph{The Name of Jesus}. Perhaps you've heard of it. It's really very fine. So I'm using his structure, but I'm going to fill in the strucutre with stories from the Desert Fathers. And at the end what we want to do is to draw some conclusions about unceasing prayer. I'm going to stick to 45 minutes however, so even if I haven't finished we'll break off and pick it up again later this afternoon.

So let's look at what the Desert Fathers have to say about unceasing prayer.

An interesting constant in this teaching is that a life of prayer requires a context. I can't just right away give you a teaching about prayer if you're not living a certain kind of Christian life. And that context is described with various words: the ascetic life, the practical life (that's the contrast in Greek between \emph{theoretikos} and \emph{praktikos}: the contemplative life, and the life of dealing with vices and virtues, the practical life). It's also just called a rule of life, and St Benedict calls it \emph{conversatio}. So in order to have a life of prayer, you have to have a rule of life, a context, a way of living. So the Fathers will talk about that first of all.

The Desert Fathers had an explicit doctrine on the means of attaining continual prayer. But in the first place they insisted on the necessity of \emph{praxis} or \emph{ascesis}. The ascetical life, or in Evagrius' terminology, \emph{praktikos}, is the spiritual method for cleansing the affective part of the soul. So the whole work of dealing with our demons, our vices, and the traditionalist's eight: gluttony, fornication, avarice, anger, sadness, acedia, vainglory, pride. In the West that comes over to the Seven Capital Sins with a slight reorganisation, because seven is a mystical number that means completeness. That happened in Gregory the Great's time. But the earlier tradition talks about eight. Dealing with those things is part of the context of the life of prayer, and acquiring the opposite virtues to these vices.

Cassian haa this to say about this context I'm talking about: ''Whoever wishes to arrive at the state of contemplation,'' the theoretical life, ''must first put all his efforts and ambition into the acquisition of practical knowledge.'' By that he means rooting out vices and acquiring virtues.

There's a monk named Dioscorus, who describes his practical life, his rule of life, and for him every year he had a particular resolution about some aspect of this practical life. One year he proposed not to go out and visit anyone, as a hermit, that is. Another year, not to speak. Another year, to eat only fruit and then not to waste vegetables. Okay, that sounds a little bit funny to us, but this is the literary genre of the Desert Fathers. But the point of this little citation from Dioscorus, is that he had a way of life. In Greek, a \emph{politeia}.

No we know, that for St Benedict, that concept is summarised by the word \emph{conversatio}. Here's just a little words study about the Greek word \emph{politeia}. \emph{Politeia} referred to the inner life of virtues and holiness together with the means and methods to achieve that goal. It was the \emph{politeia}, or in Latin the \emph{conversatio} that made the saint. So when St Benedict describes \emph{conversatio morum}, through the centuries that's filtered to become an explicit vow, althought in St Benedict's time it wasn't considered in that juridical fashion, nonetheless, that vow that we make of \emph{conversatio} means this context for a life of prayer. That's where all that stuff comes from.

Let's see what the Father's have to say, once this context is established --- for us it's just the monastic life --- let's see what they have to say about continuous prayer.

Let's start with Origen, who's not exactly one of the Desert Fathers, but nonetheless it's a good place to start. Origen has a treatise on prayer, in which he says, ``The man who prays continually is the man who combines prayer with necessary works and combines works with prayer. This is the only way it seems possible to fulfill the precept of unceasing prayer. We have to envision the whole life of a pious Christian as one long prayer. And the exercise we commonly refer to as prayer is merely a part of this whole.'' That is, there's a distinction being made between explicit moments of prayer and implicit prayer. So I'd like to divide my remarks into those two categories. What do the Desert Fathers tell us about explicit prayer and what do they tell us about a state of prayer, or implicit prayer. Those are the two categories we'll be using.

Under explicit prayer, first of all, I'd like to talk about specific actions that people perform, beginning with gestures. The Desert Fathers didn't pray only using what we call nowadays mental prayer. But they also prayed with their body, with gestures. Let me begin with our friend Arsenius again. ``It was said of him that on Saturday evenings, preparing for the glory of Sunday, he would turn his back on the Sun,'' that is, he would face East, with his back to the West, to the setting sun, ``and stretch out his hands in prayer towards the heavens until once again the sun shone on his face,'' next morning, ``then he would sit down.'' So all night long with arms stretched out --- now that's a torture! People use that as a torture, you know, to have somebody stand like this for hours or at least for as long as they could --- anyway, here's this heroic Arsenius who prays with arms stretched out all night long till the morning sun, which symbolises the resurrection. When we pray, at least most of us, because of the culture of the last couple hundred years, we tend to have a kind of intellectual view of prayer, that is, prayer is something that you do with your head, and we tend to forget the rest of our body. But praying with arms outstretched, just as the priest prays at Mass, in the early church that was a very common way of praying. You see examples of that in the catacombs, of people praying with their arms raised to heaven that way.

There's also and example of a monk named Tithowez. If you're looking for a good monastic name to suggest to people, the Desert Fathers have some very interesting ones: Brother Tithowez. ``It was said of Abba Tithowez that when he stood up to pray, if he did not quickly lower his hand, his spirit was rapt up to heaven. So if it happened that some brothers were praying with him, he hastened to lower his hands so that his spirit would not be rapt and he should not pray for too long.'' Because he didn't want to go into an ecstasy in front of the other monks, so whenever he would raise his hands he would go into ecstasy, so he would make sure to keep his hands down so that  the other monks wouldn't see that he was a holy man. So praying with hands outstretched is one of these examples of action of prayer.

There's also the question of what the Greeks called ``making metanies'', or genuflections or prostrations. And here's an example from Simon the Stylite in Syria. Now you know of his rather curious way of life, that he lived on a great pillar and there was a platform up there and he lived up on the pillar. So here's description of how Simon would pray: ``At times he would remain standing motionless and at other times he would bow down low again and again to offer God his adoration. The spectators would often count these adorations. Once a companion of mine counted 2144, then he got mixed up and stopped counting.'' Well, of course, those are feats that we think are just outlandish and perhaps impossible, but the point is, that part of his prayer was making prostrations or genuflections.

Here's another story that emphasises the same thing. This is a Roman monk named Christopher, but he lived in Palestine. He describes his way of prayer. ``By day, I would busy myself as the rule prescribed and by night I would withdraw to the grotto which holds the tomb of St Theodosios,'' the founder of that monastery, ``and the other fathers in order to pray there. As I descended there I made a hundred genuflections to God on each of the eighteen steps.'' So there's eighteen steps and a hundred genuflections on each step, that means 1800 genuflections. That would go on all night long! ``And when I had gone down all the steps I stayed there until the bell rang. Then I went up to attend the Office.'' Once again, a kind of larger than life description of physical prayer, but perhaps the very fact that it's larger than life might inspire us to do some little, small thing, to pray not only with our thoughts , with our mind, but also with our body. It can help a great deal because as you know there are lots of times when we don't feel like praying. And even if we try and discipline our mind, our mind is like a vagabond, it goes hither and yon, and you can't get it to concentrate. Well, the body can help the mind. If the mind is just no good, and sometimes it's not: you're tired, you've had a lot to do, you can't think any more, well you don't have to think necessarily when you pray! You can just raise your hands to God or make a genuflection or prostration --- in the privacy of your cell of course, so you don't cause \emph{admiratio} with the other confreres. But it's important to know that part of the rule of life or \emph{conversatio} of these Desert Fathers was physical prayer.

They also prayed using words. Here's another example of our buddy Arsenius. ``Abba Arsenius was in the habit of whispeing to himself the words, `Arsenius, why have you come here?' in other words, `For what purpose have you left the world?' He also had the practice of singing to himself this refrain: `I have often had regrest for speaking but never for keeping silence.''' The point is that he would pray whispering to himself. When we first got to Norcia, the Basilica of St Benedict had been run by one of the parish priests, and so it was a place for people to come and make their devotions. And there was this older woman who was a little bit retarded and she would come in and pray, like at the altar of St Benedict and the altar of St Scholastica, and she always prayed out loud. And of course, for the monks that was a little bit disturbing, because we would prefer that people pray silently so that we could pray. But she couldn't pray without saying something. And she would mumble her prayers and then go. Well, that's very instructive because in fact that's how the ancients prayed. Out loud. At least, very often.

One of the staples of monastic prayer was psalmody. But the Fathers, and St Benedict too, make a distinction between the prayer and psalmody. That is, reciting psalms is not the same to them as \emph{oratio}. Those are different kinds of operations. Let's see what the say about psalmody. Now, we're still talking about the Desert Fathers, not organised community life, so what organised communities do with the psalms is different from the hermits of the desert. Here's a story of Epiphanius who was bishop of Cyprus, so we're not in Egypt, we're not in Palestine, we're not in Syria, we're not in Asia Minor, we're on one of the islands there. But it's all part of the same experience. ``The abbot of this cenobium in Palestine wrote to Epiphanius bishop of Cyprus saying, `Thanks to your prayers we have been faithful to our Canonical Hours. We never omit the Office of Terce, Sext, None, or Vespers.''' So they had four Canonical Hours in that monastery. Origen, earlier, will talk about three, that is Terce, Sext, and None. We'll see that the Canonical Hours develop with time in order to arrive at seven in the Rule of St Benedict, but in this case it's a bit earlier than that. And he's proud, this abbot, that in his monastery they're faithful to these four Canonical Hours. ``But the bishop wrote back and he reproached the monks in these terms, `Evidently you are neglecting the remaining hours of the day which you spend without prayer. The true monk should have prayer and psalmody in his heart at all times, without interruption.''' So once again, the ideal is prayer without ceasing, making a distinction between prayer and psalmody. The psalmody at the Canonical Hours provides a kind of skeleton. But if that's all the prayer that we do, that is the saying of psalms, the Epiphanius says you're not really praying.

Here's another quote from Epiphanius, ``The same old man said, `David the Prophet prayed late at night.''' Let's try and count these. They will become the Canonical Hours: late at night, in the middle of the night somehow, ``waking in the middle of the night he prayed before the day,'' like at dawn, ``at the dawn of day he stood before the Lord. In the small hours he prayed,'' that would be Terce, Sext and None. ``In the evening and at midday,'' the order is a little bit strange, ``he prayed again. And this is why he said, \emph{seven times a day have I praised you}.'' In this particular citation the seven times of day are the middle of the night, which we would call Vigils or Mattins, dawn, which we would call Lauds, the little hours, Terce, Sext and None, evening, which would be Vespers, and late at night, which we would call Compline. What's missing from there is Prime, which develops just a little bit later. In any case, the use of psalmody in the Divine Office as part of the way of life, the context for continuous prayer is a basic ingredient of the Desert Fathers. So the psalms have as their purpose to help the monk in a life of continuous prayer.

There's good psychological insight here. We can't concentrate all the time, even on the things of God, because physically our human body is made in such a way that you just can't do that. You have to take a break every now and again and alternate one thing with another. There's a story about Anthony of the Desert, who was preoccupied with this question and he didn't know what to do and prayed to God for enlightenment and he had a vision of an angel in his cell. And the angel sat down and started weaving some baskets and then after a while he got up and prayed. And then after a while he sat down again and wove some more baskets. And after a while he got back up and prayed. And the angel said to him, ``Do this and you will be saved.'' So Anthony learned then you have to alternate the explicit prayer with other things, with other kinds of activity. We see this also in a saying of John the Dwarf. ``It was said of the same Abba John that when he returned from the harvest'' -- so the monks hired themselves out to the local farmer at harvest time. That's why St Benedict says once the monks have to do the harvest themselves, okay, comma, \emph{but!}, because it's a little bit disruptive to the common life. In any case, ``When he returned from the harvest or when he had been with some of the old men,'' to visit them, that is, ``he gave himself to prayer, meditation and psalmody, until his thoughts were reestablished in their previous order.'' We'll look more closely at what prayer and meditation means, but here he's alternating prayer and meditation and psalmody in order to reestablish his thoughts. Why? Because if you go out you often dissipate it in a certain way. I was feeling this very keenly after a long trip that you're just going through airports and sitting in the plane and hearing all this music and it's all very distracting. And it prevents you from concentrating. You're whole focus is lost. And you need to come to a monastery like this to get your thoughts back together, to reestablish your interior concentration in its previous order. So John the Dwarf alternated prayer, mediatation and psalmody.

There's also a story about someone named Lucius. This is in the context of a heresy at that time of the so-called Massalians, who wanted to pray at all times without ceasing so they wouldn't do any work. They were also called Euchites, from \emph{euche} which means to pray. ``Some of the monks who were called Euchites went to see Abba Lucius. The old man said to them, `What is your manual work?' They said, `We don't touch manual work, but as the Apostle says, we pray without ceasing.' The old man asked them if they did not eat. They replied, they did. So he said to them, `When you are eating, who prays for you then?' Again he asked them if they didn't sleep. And they replied that they did. And he said to them, `When you are asleep, who prays for you then?' They could not find any answer to give him. He said to them, `Forgive me, but you do not act as you speak. I will show you how, while doing my manual work, I pray without ceasing. I sit down with God, soaking my reeds and plaiting my ropes and I say, \emph{God, have mercy upon me, according to your great goodness, and according to the multitude of your mercies,}''' the first line of Psalm 50, ```and save me from my sins.' So he asked them if this were not prayer. And they replied, it was. Then he said to them, `So, when I spent the whole day working and praying, making thirteen pieces of money more or less, I put two pieces of money outside the door and I pay for my food with the rest of the money. He who takes the two pieces of money,''' that is, for the poor, ```prays for me when I am eating and when I am sleeping. So by the grace of God I fulfill the precept to pray without ceasing.''' Well, that's a little bit materialistic, you know, that you have to measure things out so precisely. But the point that I'm trying to make here is that he alternated work and prayer. Work and psalmody. That was a common practice then of the Desert Fathers.

In addition to these actions or these acts of prayer, gestures, verbal prayer, psalmody, the alternation of prayer and work, there's also something called, as part of explicit prayer, ``secret meditation''. And I'd like to look at this in just a little bit.

In addition to exterior work, the Desert Fathers had this concept of interior activity. And the interior activity had to do not only with dealing with your thoughts, with these eight vices that I just mentioned earlier, but also engaging your mind in prayer. And this was done not just in a mental way, but in a verbal way. A verbal way of praying, or of repeating the Scriptures, or of memorising the Scriptures. This practice is not unique to the Desert Fathers, or to Christians, but in fact comes from the Greek philosophical tradition. Because in certain Greek schools of philosophy, the students who wanted to be philosophers were trained in the \emph{soliloquy}, which was a kind of training, of training oneself (soliloquy is only something you say, it's not a dialogue), a method of training oneself by the spoken word or by writing. It consists primarily in saying over and over to oneself, either quietly or more loudly, certain sentences which the student wishes to engrave on his memory. So, memorisation. Soliloquy in the Greek philosophical tradition required memorising things. Well, the Desert Fathers picked that up and used that method for memorising large parts of the Scriptures. When we talk about Lectio Divina we'll talk more about that too. But in terms of some of these annecdotes of the Fathers, about how they prayed, by ruminating just as a cow does












