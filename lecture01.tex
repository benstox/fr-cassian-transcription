\chapter{Praying without ceasing}

\pagestyle{fancy}

In the name of the Father and of the Son and of the Holy Spirit.

Let us pray.

Pour forth, we beseech Thee, O Lord, Thy Grace into our hearts, that we, to whom the Incarnation of Christ, Thy Son, was made known by the message of an angel, may through His Passion and Cross be brought to the glory of His Resurrection. Through the same Christ our Lord.

Well, I'm very happy to be with you again. As we were talking about last night at recreation, there's a long history of connection and friendship which I value very much. It seems strange perhaps that monks need a retreat, but in fact our lives are very busy and rather intense, so a retreat is a wonderful thing in which you can respond to the Lord's invitation to \emph{come away to a quiet place and rest a while}. So I'm sure getting some extra rest is important this week, but also a time to reflect, a kind of review of the past year. And we tend to ask ourselves questions like, where am I in the spiritual life? And what is God doing in my life? What are my preoccupations? What changes do I need to make? Maybe I need to reorient myself, to get back on track. How is my life of prayer? The question that you need to ask God will give you to ask, but we need to ask them.

The topic of the retreat is prayer. It's a general theme suggested by Abbott Xavier in an e-mail to me. He was talking about the relationship between Mass and Office in particular, if I remember correctly. And I was thinking about that, and thought, well, that's crucial but it has to fit into a larger picture, because there's also Lectio, there's private prayer, there's all kinds of other things. We're searching for unity in our spiritual life, not fragmentation. So what I'd like to do, or try to do, is to provide that unifying context in which we can examine more particularly the various aspects of our life of prayer. That's why I gave you this little outline here, so that you'll know where we're going.

The question of prayer, of course, is basic to our life and it's a response to Our Lord. We call the disciples' request of Jesus, ``Lord, teach us how to pray''.  It's very curious the way he goes about responding to that question. And there's a little icon, which describes the scene. A very beautiful one. I'll just pass it around --- show-and-tell --- so you have something to do. But this icon is also in the third part of the Catechism in the section on prayer. That is, each of those four sections has some kind of image from the tradition which summarizes the whole content of that section. And this is the very image that's in the Cathechism in the section on prayer.

The disciples ask Jesus to teach them to pray, so you'd think that he would turn toward them and give them some sort of instruction, and engage them in some sort of dialogue. But he doesn't. In the icon he's not facing the disciples at all. He turns and faces the Father. The Father is pictured up in a little cloud, because the cloud is the symbol of divinity. And he addresses himself to the Father. And the disciples observe that. You can see in the icon they're watching to see what's going on there. And it's a perfect definition of prayer: that is, not something that we do, in the first place, but somehting that we participate in. Prayer is the dialogue between the Father and the Son, and the communion of the Holy Trinity. And we are invited in to that. It's going on quite independently of us. So Whether we pray or not, the Son is praying to the Father. And so, just as in the liturgy, it's not so much something that we do, but the heavenly liturgy, in which we participate. So also in other kinds of prayer, it's the key relationship between the Father and the Son to which we are invited.

So monastic prayer is nothing more than a deepening of this Christian prayer. The fact is that we dedicate time and energy and the orientation of our whole life to prayer. Cassian describes this preoccupation of the monk with the things of God and our desire to participate in the prayer between Jesus and the Father in these words: ``The goal is that we shall see the fulfillment of our Saviour's prayer to the Father for his disciples''. And he quotes John's Gospel, ``May the love with which you have loved me,'' Jesus says to the Father, ``be in them and they in us. May they be one in you, Father, as you are in me and I am in you''. That they also may be one in us. So this communion with the Trinity is the goal of our prayer. This perfect love with which God has first loved us will come about in our hearts through the fulfillment of this prayer of the Lord. And our faith tells us that his prayer cannot be in vain. So what's key is the prayer of `Christ. These are the signs that will accompany the fulfillment, the signs in us. God will be our whole love and desire. Our whole study and labour. Our whole thought, our whole life. Our speaking. Our breathing. The unity which exists between the Father and the Son will be communicated to our souls and emotions. Just as God loves us with a true and pure and unfailing charity, so we will be united to him with the indissoluble unity of constant love. We will be so attached to him that our whole yearning and thinking and speaking will be about him alone. So the life of prayer to which we aspire is something that embraces us in a total kind of way.

I'd like to being with  the desert tradition because the retreat is entitile Prayer in the Monastic Tradition and the Desert Tradition is where we get the beginnings of this kind of theology of preayer that will be developed in the later tradition. Now the Desert Tradition doesn't only mean Egypt, it also includes Palestine and Syria and Asia Minor. It's that whole area. But anyways, those first few centuries of monastic experience.

The Desert Fathers were really concerned with the Gospel injunction to pray without ceasing. In Luke 18, the parable about the pesty widow who molests the judge with her constant petitions, the beginning of that parable says, the Lord gave the parable so that we would learn to pray without ceasing and never give up. Also in 1 Thessalonians 5:17, St Paul lists a number of desirable qualities: \emph{rejoice always and pray without ceasing}. We know that the Desert Fathers took this seriously, and there are some wonderful stories that describe this, so I'll be telling a lot of stories in this first conference. They're more interesting anyway. First, about Arsenius: ``Having withdrawn to the solitary life, Arsenius made this prayer: `Lord, lead me in the way of salvation'. And he heard a voice saying to him, `Arsenius, flee, be silent, pray always, for these are the sources of sinlessness'''. So part of the vocation of Arsenius is to pray without ceasing. There's another Desert Father, Benjamin, who talks about the same thing. ``As he was dying, Abba Benjamin said to his sons, `If you observe the following you can be saved,''' and he quotes 1 Thessalonians 5, ```\emph{be joyful at all times, pray without ceasing, and give thanks for all things}'''.

This ideal of praying without ceasing is very attractive, but how can we achieve it? It's one thing to present the ideal, it's another thing to get there. We're familiar with the story of the Russian pilgrim, \emph{The Way of the Pilgrim}, which popularised the Jesus prayer, and you might recall that in the beginning of this pilgrim's search he was looking for a wise old man who could tell him how to pray without ceasing. And he heard that there was a certain hermit living on a country estate, and he went and found him and said, ```I beg you in God's name to explain to me the hidden meaning of the Apostle's words, \emph{pray without ceasing}. How is it possible to pray continually?' The man gazed at me for a long time before speaking.'' There's a certain reluctance to communicate these secrets. In Casssian's report in Conferences 9 and 10 where Abba Isaiah is the one speaking, Cassian and Germanus, his companion in crime, ask Abba Isaiah to teach them how to pray. And he says, well, I'm really not sure I want to tell you. I'm paraphrasing of course. But the idea is that you have to be serious about it. You can't throw you pearls before the swine. If somebody is not serious, I'm not going to tell you. There's no point in bothering about it. So, there's a kind of theme, a constant theme in the tradition, of reluctance to explain these intimate things about our relationship with God. For us that's rather strange. We think of evangelisation and the need to preach and give them their method and all that sort of thing. The Fathers are a bit more reserved. And there's a wisom to that too. If you're serious about this, then it'll cost you something.

''So the old man said to the pilgrim, 'Continual interior prayer is the spontaneous leaping forth of man's spirit toward God.''' Now he's not going to give him a method right away. He says, '''You have to pray to God to enlighten you about the means by which the Spirit may accomplish this activity. Pray hard and fervently. Prayer itself will reveal to you how it can be prolonged unceasingly. But this will take much time.''' So there's no magical recipe for prayer, and we learn about prayer by praying. Those are good principles.

Now in the description I'm going to give about the Desert Teaching, I'm going to use a systematic framework that was provided by a Patristic scholar named Iren\'{e} Orsaire, who has a little book, well, not so little, called \emph{The Name of Jesus}. Perhaps you've heard of it. It's really very fine. So I'm using his structure, but I'm going to fill in the strucutre with stories from the Desert Fathers. And at the end what we want to do is to draw some conclusions about unceasing prayer. I'm going to stick to 45 minutes however, so even if I haven't finished we'll break off and pick it up again later this afternoon.

So let's look at what the Desert Fathers have to say about unceasing prayer.

An interesting constant in this teaching is that a life of prayer requires a context. I can't just right away give you a teaching about prayer if you're not living a certain kind of Christian life. And that context is described with various words: the ascetic life, the practical life (that's the contrast in Greek between \emph{theoretikos} and \emph{praktikos}: the contemplative life, and the life of dealing with vices and virtues, the practical life). It's also just called a rule of life, and St Benedict calls it \emph{conversatio}. So in order to have a life of prayer, you have to have a rule of life, a context, a way of living. So the Fathers will talk about that first of all.

The Desert Fathers had an explicit doctrine on the means of attaining continual prayer. But in the first place they insisted on the necessity of \emph{praxis} or \emph{ascesis}. The ascetical life, or in Evagrius' terminology, \emph{praktikos}, is the spiritual method for cleansing the affective part of the soul. So the whole work of dealing with our demons, our vices, and the traditionalist's eight: gluttony, fornication, avarice, anger, sadness, acedia, vainglory, pride. In the West that comes over to the Seven Capital Sins with a slight reorganisation, because seven is a mystical number that means completeness. That happened in Gregory the Great's time. But the earlier tradition talks about eight. Dealing with those things is part of the context of the life of prayer, and acquiring the opposite virtues to these vices.

Cassian haa this to say about this context I'm talking about: ''Whoever wishes to arrive at the state of contemplation,'' the theoretical life, ''must first put all his efforts and ambition into the acquisition of practical knowledge.'' By that he means rooting out vices and acquiring virtues.

There's a monk named Dioscorus, who describes his practical life, his rule of life, and for him every year he had a particular resolution about some aspect of this practical life. One year he proposed not to go out and visit anyone, as a hermit, that is. Another year, not to speak. Another year, to eat only fruit and then not to waste vegetables. Okay, that sounds a little bit funny to us, but this is the literary genre of the Desert Fathers. But the point of this little citation from Dioscorus, is that he had a way of life. In Greek, a \emph{politeia}.

No we know, that for St Benedict, that concept is summarised by the word \emph{conversatio}. Here's just a little words study about the Greek word \emph{politeia}. \emph{Politeia} referred to the inner life of virtues and holiness together with the means and methods to achieve that goal. It was the \emph{politeia}, or in Latin the \emph{conversatio} that made the saint. So when St Benedict describes \emph{conversatio morum}, through the centuries that's filtered to become an explicit vow, althought in St Benedict's time it wasn't considered in that juridical fashion, nonetheless, that vow that we make of \emph{conversatio} means this context for a life of prayer. That's where all that stuff comes from.

Let's see what the Father's have to say, once this context is established --- for us it's just the monastic life --- let's see what they have to say about continuous prayer.

Let's start with Origen, who's not exactly one of the Desert Fathers, but nonetheless it's a good place to start. Origen has a treatise on prayer, in which he says, ``The man who prays continually is the man who combines prayer with necessary works and combines works with prayer. This is the only way it seems possible to fulfill the precept of unceasing prayer. We have to envision the whole life of a pious Christian as one long prayer. And the exercise we commonly refer to as prayer is merely a part of this whole.'' That is, there's a distinction being made between explicit moments of prayer and implicit prayer. So I'd like to divide my remarks into those two categories. What do the Desert Fathers tell us about explicit prayer and what do they tell us about a state of prayer, or implicit prayer. Those are the two categories we'll be using.

Under explicit prayer, first of all, I'd like to talk about specific actions that people perform, beginning with gestures. The Desert Fathers didn't pray only using what we call nowadays mental prayer. But they also prayed with their body, with gestures. Let me begin with our friend Arsenius again. ``It was said of him that on Saturday evenings, preparing for the glory of Sunday, he would turn his back on the Sun,'' that is, he would face East, with his back to the West, to the setting sun, ``and stretch out his hands in prayer towards the heavens until once again the sun shone on his face,'' next morning, ``then he would sit down.'' So all night long with arms stretched out --- now that's a torture! People use that as a torture, you know, to have somebody stand like this for hours or at least for as long as they could --- anyway, here's this heroic Arsenius who prays with arms stretched out all night long till the morning sun, which symbolises the resurrection. When we pray, at least most of us, because of the culture of the last couple hundred years, we tend to have a kind of intellectual view of prayer, that is, prayer is something that you do with your head, and we tend to forget the rest of our body. But praying with arms outstretched, just as the priest prays at Mass, in the early church that was a very common way of praying. You see examples of that in the catacombs, of people praying with their arms raised to heaven that way.

There's also and example of a monk named Tithowez. If you're looking for a good monastic name to suggest to people, the Desert Fathers have some very interesting ones: Brother Tithowez. ``It was said of Abba Tithowez that when he stood up to pray, if he did not quickly lower his hand, his spirit was rapt up to heaven. So if it happened that some brothers were praying with him, he hastened to lower his hands so that his spirit would not be rapt and he should not pray for too long.'' Because he didn't want to go into an ecstasy in front of the other monks, so whenever he would raise his hands he would go into ecstasy, so he would make sure to keep his hands down so that  the other monks wouldn't see that he was a holy man. So praying with hands outstretched is one of these examples of action of prayer.

There's also the question of what the Greeks called ``making metanies'', or genuflections or prostrations. And here's an example from Simon the Stylite in Syria. Now you know of his rather curious way of life, that he lived on a great pillar and there was a platform up there and he lived up on the pillar. So here's description of how Simon would pray: ``At times he would remain standing motionless and at other times he would bow down low again and again to offer God his adoration. The spectators would often count these adorations. Once a companion of mine counted 2144, then he got mixed up and stopped counting.'' Well, of course, those are feats that we think are just outlandish and perhaps impossible, but the point is, that part of his prayer was making prostrations or genuflections.

Here's another story that emphasises the same thing.













