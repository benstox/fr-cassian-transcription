\chapter{Praying without ceasing}

\begin{itshape}
Raise up in your Church the Spirit with which our Holy Father St Benedict was animated, so that filled with the same Spirit, we may strive to love what he loved and practise what he taught. We ask this through Christ our Lord. Amen.
\end{itshape}

\smallskip

So we continue our exploration of prayer without ceasing in the Desert Tradition, and then we're going to try and connect that with the Benedictine tradition.

We spoke about explicit prayer and implicit prayer. I'd like to begin with the notion of implicit prayer or a state of prayer. We know from experience that we can't always be praying. We can't always be saying something or thinking something. So to pray without ceasing must mean something more than that. And so perhaps it means a state of prayer which sustains individual moments of prayer.

Cassian uses the term \emph{orationis status}, the state of prayer, and the author that I mentioned to you this morning, Ir\'{e}n\'{e}e Hausherr, he explores the meaning of the Greek term for ``state'' and says that it means putting things in order. It refers to harmony, permanence, rest, firmness in a condition conformed to a thing's nature, a thing's laws and perfection. I like the notion of putting things in order and being in the state that corresponds to your nature. Now, a state can't be something transitory. It can't be isolated events that take place. It has the characteristics rather of a habit or disposition. We could talk about the habit of prayer or the disposition of prayer.

This is the way one of the Desert Fathers describes the habit of prayer or the state of prayer, using the example of wine. ``Wine first goes through a stage of fermentation. This symbolises the age of adolescence, which is a period of excitement, inevitable until maturity and steadiness are attained. Wine does not reach maturity unless the proper amounts of lime and yeast are added and in the same way it is impossible for youth to advance in the practice of self denial unless it receives the proper leaven from a spiritual father in the form of guidance along the right path until God's grace gives him clear vision. People leave wine in a cellar or in a cave until it is perfectly settled.'' --- and the word for settled is the Greek word meaning the state --- ``Likewise, without austerity, \emph{hesychia}'' --- that is stillness --- ``and all sorts of good works, it is not possible to attain to this state.'' So a new wine, newly fermented, needs time to rest and mature to acquire the quality of a fine wine, and the resting of the wine is like a state. And so also in our life of prayer, the initial fervour of youth is wonderful and perhaps we look back on it (well, some of us look back, you look forward or are in it now) as a time of great joy and excitement. But that has to mature and become more profound so that it becomes a state, not just a passing moment of enthusiasm.

The tradition speaks of this state of prayer in other terms. I'd like to single out the whole approach of the ``remembrance of God''. Some spiritual authors, even in the Desert Tradition, have a kind of intellectual view of prayer and are looking for pure prayer and try to eliminate all sorts of distractions and so on, so as to focus and concentrate the mind. But this pure prayer, a state of pure intellectuality (perhaps we could put Evagrius in that category), the vision of the Holy Trinity as it is described, all of that is extremely rare. And if that's the case, then the vast majority of men, even the majority of monks, must give up all hope of attaining union with God in the state of prayer. It seems a little bit too refined. Most of us are kind of ordinary. But in the monastic tradition, in place of such lofty ambitions, the Fathers propose striving for a state which may be more lowly in appearance but is perhaps not inferior in excellence. A state which is generally far more attainable, though it does not lack difficulties of its own. The traditional name for this state is the ``memory of God''. Because that's more accessible, that's what I'd like to focus on.

Even in the pre-Christian philosophical tradition of Philo there's an effort to emphasise preserving the memory of God and never forgetting it. And that phrase sounds very much like the Rule of St Benedict, Chapter 7, on the first degree of humility: ``to flee forgetfulness''. St Benedict puts that at the beginning of this journey of humility, that is to be aware of the presence of God and not to forget it. In fact, to \emph{flee} forgetfulness! You can turn the steps of humility the other way around too and say this is the summmit, as opposed to the twelfth degree, that is to be aware of the presence of God at all times. The memory of God at all times. St Benedict doesn't give a systematic treatment of the spiritual life but every now and then there are little hints in the Rule that refer to a whole teaching of the larger tradition and that gives us some insight into what he's talking about. So when he says to flee forgetfulness, that's really the whole theology of the memory of God. Let's see what else the Fathers say about this memory of God.

St Basil talks about this and describes as the goal






