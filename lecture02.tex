\chapter{Praying without ceasing}

\begin{itshape}
Raise up in your Church the Spirit with which our Holy Father St Benedict was animated, so that filled with the same Spirit, we may strive to love what he loved and practise what he taught. We ask this through Christ our Lord. Amen.
\end{itshape}

\smallskip

So we continue our exploration of prayer without ceasing in the Desert Tradition, and then we're going to try and connect that with the Benedictine tradition.

We spoke about explicit prayer and implicit prayer. I'd like to begin with the notion of implicit prayer or a state of prayer. We know from experience that we can't always be praying. We can't always be saying something or thinking something. So to pray without ceasing must mean something more than that. And so perhaps it means a state of prayer which sustains individual moments of prayer.

Cassian uses the term \emph{orationis status}, the state of prayer, and the author that I mentioned to you this morning, Ir\'{e}n\'{e}e Hausherr, he explores the meaning of the Greek term for ``state'' and says that it means putting things in order. It refers to harmony, permanence, rest, firmness in a condition conformed to a thing's nature, a thing's laws and perfection. I like the notion of putting things in order and being in the state that corresponds to your nature. Now, a state can't be something transitory. It can't be isolated events that take place. It has the characteristics rather of a habit or disposition. We could talk about the habit of prayer or the disposition of prayer.

This is the way one of the Desert Fathers describes the habit of prayer or the state of prayer, using the example of wine. ``Wine first goes through a stage of fermentation. This symbolises the age of adolescence, which is a period of excitement, inevitable until maturity and steadiness are attained. Wine does not reach maturity unless the proper amounts of lime and yeast are added and in the same way it is impossible for youth to advance in the practice of self denial unless it receives the proper leaven from a spiritual father in the form of guidance along the right path until God's grace gives him clear vision. People leave wine in a cellar or in a cave until it is perfectly settled.'' --- and the word for settled is the Greek word meaning the state --- ``Likewise, without austerity, \emph{hesychia}'' --- that is stillness --- ``and all sorts of good works, it is not possible to attain to this state.'' So a new wine, newly fermented, needs time to rest and mature to acquire the quality of a fine wine, and the resting of the wine is like a state. And so also in our life of prayer, the initial fervour of youth is wonderful and perhaps we look back on it (well, some of us look back, you look forward or are in it now) as a time of great joy and excitement. But that has to mature and become more profound so that it becomes a state, not just a passing moment of enthusiasm.

The tradition speaks of this state of prayer in other terms. I'd like to single out the whole approach of the ``remembrance of God''. Some spiritual authors, even in the Desert Tradition, have a kind of intellectual view of prayer and are looking for pure prayer and try to eliminate all sorts of distractions and so on, so as to focus and concentrate the mind. But this pure prayer, a state of pure intellectuality (perhaps we could put Evagrius in that category), the vision of the Holy Trinity as it is described, all of that is extremely rare. And if that's the case, then the vast majority of men, even the majority of monks, must give up all hope of attaining union with God in the state of prayer. It seems a little bit too refined. Most of us are kind of ordinary. But in the monastic tradition, in place of such lofty ambitions, the Fathers propose striving for a state which may be more lowly in appearance but is perhaps not inferior in excellence. A state which is generally far more attainable, though it does not lack difficulties of its own. The traditional name for this state is the ``memory of God''. Because that's more accessible, that's what I'd like to focus on.

Even in the pre-Christian philosophical tradition of Philo there's an effort to emphasise preserving the memory of God and never forgetting it. And that phrase sounds very much like the Rule of St Benedict, chapter 7, on the first degree of humility: ``to flee forgetfulness''. St Benedict puts that at the beginning of this journey of humility, that is to be aware of the presence of God and not to forget it. In fact, to \emph{flee} forgetfulness! You can turn the steps of humility the other way around too and say \emph{this} is the summmit, as opposed to the twelfth degree, that is to be aware of the presence of God at all times, the memory of God at all times. St Benedict doesn't give a systematic treatment of the spiritual life but every now and then there are little hints in the Rule that refer to a whole teaching of the larger tradition and that gives us some insight into what he's talking about. So when he says to flee forgetfulness, that's really the whole theology of the memory of God. Let's see what else the Fathers say about this memory of God.

St Basil talks about this and describes as the goal the ``holy thought of God stamped into our soul like an ineradicable seal'' --- a stamp --- ``by means of a distinct and continual remembrance.'' For St Basil it is perfectly possible to conserve this remembrance in all circumstances, even while doing physical labour. So occupations, for St Basil, outside the time of formal prayer are perfectly compatible with prayer of the heart or even of the lips. They provide opportunities for such prayer, stimulated, and so to speak prepare for it. Well, the Desert Fathers certainly did that with their manual labour, that is, so that they could be working with their hands and be repeating psalm verses or this One-Word Prayer at the same time, in order to keep the memory of God even while going about other occupations. For intellectual work that's a bit more difficult, because if you have to study, then your mind has to be focussed on what it is that you're studying and it's harder to keep in mind the memory of God unless it has become a habit or a state as I was saying before.

Let me give a little summary then of what we've said so far. In approaching the Desert Fathers the question I raised is what is necessary for continuous prayer. They all say that they want it. How do we go about it? One of the requirements that we saw was \emph{conversatio}, that is, a way of life or rule of life. A certain ascetic discipline. That's the context in which prayer takes place. A unity between your life of prayer and your other kinds of activity. Then we spoke about explicit moments of prayer or kinds of prayer: gestures, verbal prayer, psalms (and the psalms will be the link with the Divine Office), the alternating of prayer and memorisation and psalmody, we spoke briefly about secret meditation, that is the interior work that is done for Lectio Divina, the rumination over a scriptural text, and this One-Word Prayer, that Cassian talks about, which I will say more about later on this week. All those are explicit kinds of prayer. Then briefly we've looked at implicit prayer, that is, what we really hope for and desire, aside from individual moments of prayer and individual styles of prayer: a habit of prayer or a state of prayer, which is like a mature wine and involves the living in the presence of God and having the memory of God always before us.

Those are wonderful ideals. How are they realised in the monastic tradition? I'd like to switch to the Divine Office now and how these monastic ideals are made concrete in the monastic Office, especially but not exclusively as indicated by St Benedict. So we're turning to prayer without ceasing in the Rule of St Benedict and most especially in the Office.

It's interesting that we would think of the Office in terms of prayer without ceasing. It's interesting I say because throughout history it's been looked at in many different ways and frequently it has been considered in the history of spirituality as a burden. As a duty. As something that you have to perform, but your spiritual nourishment is really elsewhere. Over and over again throughout the centuries we've seen that kind of mistaken attitude towards the Office, whereas if we go back to the origins, the Office seems to be a continuation or a making more concrete of this desire to pray without ceasing. The way that that's revealed in the Rule of St Benedict is through symbolic numbers on the one hand and through the \emph{Deus in adjutorium} on the other.

Let me say something about symbolic numbers. You'll find this perhaps odd, but maybe curious enough to be interesting. The ancients liked numbers. And numbers had a symbolic value for them and communicated a great deal more to them than they do to us. And two numbers that are especially significant are seven and twelve. So let's look at the number seven.

As we find it in the Rule of St Benedict, this is chapter 16, the celebration of the Divine Office during the day: ``The prophet says, \emph{Seven times a day have I praised you}. We will fulfill the sacred number of seven if we satisfy our obligations of service'' --- now St Benedict considers the Office an obligation also, the ``officia nostr\ae\ servitutis'' --- ``We will fulfill the sacred number of seven if we satisfy our obligations of service at Lauds, Prime, Terce, Sext, None, Vespers and Compline, for it was of these hours during the day that he said: \emph{Seven times a day have I praised you}. Concerning Vigils, the same prophet says,'' --- we're talking about the psalms, so it's David --- ``\emph{At midnight I arose to give you praise}.'' So that's something separate outside of the number seven in St Benedict's arrangement here. ``Therefore we should praise our Creator for his just judgements at these times.'' Then he repeats himself: ``Lauds, Prime, Terce, Sext, None, Vespers and Compline, and let us arise at night to give him praise.'' Why does St Benedict make a big deal out of this? Well, because seven means totality, completeness, continuous. So if we want to pray without ceasing we have to pray seven times a day. Now you might say, well that's a kind of material way of looking at things, and not very convincing. Well, from an intellectual point of view perhaps it's not very convincing, but from a symbolic point of view it meant a lot to the ancients. Seven times a day means praying without ceasing. Not just the Rule of St Benedict but all kinds of Patristic texts --- I'm thinking of St Gregory in particular, but Augustine and others --- use the number seven in this sense. So the structure of the Office in the Rule of St Benedict is connected to prayer without ceasing by the number seven.

Also by the number twelve. In the Rule, chapter 10. Now the liturgical code in the Rule, for most people, is a little bit dry and dusty. Not too much spiritual juice there. But if you read it carefully there's all sorts of hidden treasures. In chapter 10, verse 3. The topic is the arrangement of the Night Office in summer. And so he talks about the distribution of the psalms and so on. ``In everything else the winter arrangement for Vigils is kept. Thus, winter and summer, there are never fewer than twelve psalms at Vigils, not counting psalms 3 and 94''. What's the big deal about twelve? Well, let's go to Cassian and find out what the big deal is.

This is from the \emph{Institutes} of Cassian, Book 2. And Cassian is describing the origins of monasticism, and for him the origins lie in the \emph{Acts of the Apostles}. But then the enthusiasm and fervour faded away and the monks then are picking up this tradition to live intensely the Apostolic life: ``However, in the early monastic tradition, the Fathers took careful consideration to lay down what the daily prayer for all monks should be. They agreed together to pass on an inheritance of peaceful devotion to their successors, safeguarded from all divisive uncertainty. For they were concerned lest disagreement arise about the daily celebration among men living the common life'' --- liturgical controversy was present then too, nothing new under the sun --- ``lest new variations sow the seeds of future error, jealousy or division. For once it happened that each individual'' --- we're talking about hermits now --- ``without thinking of the weakness of others, made decisions based on his own fervour about the method of prayer he thought ideal for his own faith and ability. He would take no account of what might be possible for the majority of brothers in general among whom, inevitably, a large proportion would be weak. This led to competition, each one attempting to decree an absurd number of psalms as his own strength of mind dictated.'' Now here it gets interesting! ``Some would opt for 50 psalms.'' A day, that is. ``Some 60. Others would even think themselves bound to surpass that number. And instead of a religious rule, they ended up with a pious competition of holy varieties until the time for the sacred celebration of Vespers would be taken up in argument.'' That is, you had hermits meeting for a little council and there was this kind of contest, who was saying the most psalms. St Benedict says in the end of the Rule, you know, we're so tepid and weak and no good today because our Fathers recited the whole Psalter in a single day. Well, that's this kind of context that he's referring to. So there was this meeting of the hermits and they were vying with one another, who was the best because they said the most psalms, and then it was Vesper time so they sat down for Vespers: ``And it happened that among those who wished to celebrate the office, someone stood up in the middle to sing psalms to the Lord. All the others sat,'' as is still the custom in Egypt, that is, you had one solo reader, and the others sat and listened. ``All the others sat and directed their attention to the words of the psalmist. He sang eleven psalms on an even tone with no pause between verses.'' You can see there's controversy there too, you know: how are we going to do these psalms? ``He sang eleven psalms on an even tone with no pause between verses, but separated by collects.'' So the custom was the psalm, then a time for silent prayer, and then the leader would have an oration to conclude that psalm and then they'd go to the next one. So this is what they did: eleven psalms and then the collect. ``The twelfth psalm he sang with an alleluia response and then suddenly vanished from the eyes of all, bringing the celebration'' --- and also the debate --- ``to an end.'' That is, the answer to the question how many psalms should we sing in an office is twelve. And this is called The Rule of the Angel which establishes the number twelve for the psalms. Now for them perhaps they met twice a day: twelve in the morning, twelve at night. I'm not quite sure. But as that's taken over into St Benedict that's twelve psalms for Mattins.

Why? Because twelve is also a mystical number meaning completion. There are Twelve Tribes of Israel. Twelve Apostles. That means the full complement of the people. Now, there's a very interesting confirmation of the value of these numbers in the Roman Canon. You know that the Roman Canon was formulated in the 300s. St Ambrose cites it in the 390s. So it's a very archaic and ancient document, touched up in succeeding centuries by various popes, by St Gregory in particular, who touched up the arrangement of the \emph{Communicantes}, that is the list of saints and the \emph{Nobis quoque} both. Now it's interesting that St Gregory did that because Gregory is concerned about the mystical value of numbers. This may be something that you've heard already, but probably not all of you, so I ask your patience when I repeat it. In the \emph{Communicantes} there's a whole long list of saints. First there's the Twelve Apostles, and then there are other saints added. Now I'm going to read them and you count them. Linus, Cletus, Clement, Sixtus, Cornelius, Cyprian, Lawrence, Chrysogonus, John and Paul, Cosmas and Damian. Twelve. You have apostles: twelve. And these other saints: twelve. How much is twelve times twelve? 144. If you want to say a huge number, then you multiply by 1000. So, 144 times 1000 is 144,000. In the Book of Revelation, the number of the elect --- from the Jews mind you, because there's a huge mass of people from the Gentiles --- the number of the elect from the Jews is 144,000. That doesn't mean, as some fundamentalists want it to mean, that you count each individual one and once you get to 144,001 you cut off the list and those people are out of it. Removed from salvation. Rather, 144,000 means an enormous multitude! Because twelve is the mystical number that means completion. And if it's multiplied by itself and then by 1000, it's just lots and lots and lots and you can't count them! That's a confirmation that the number twelve is significant, meaning completion. Fullness. Or in our case, without ceasing.

Now let's look at the \emph{Nobis quoque} also, with its list of saints. So, the prayer asks that \emph{partem aliquam et societatem donare digneris, cum tuis sanctis Apostolis et Martyribus}. Okay. \emph{Cum Joanne}. Now which John is that? That's John the Baptist, because John the Baptist is the Patron Saint of the Lateran. And the Feast of St John the Baptist on the 24th of June, even in the earliest times of the Roman Church, had a great deal of importance. So John the Baptist is at the head of this list as the \emph{capo}, a kind of chief, and then the other saints follow under his direction. There are male saints and there are female saints. Count the male saints: \emph{Stephano, Matthia, Barnaba, Ignatio, Alexandro, Marcellino, Petro}. Seven. What about the female saints? \emph{Felicitate, Perpetua, Agatha, Lucia, Agnete, Caecilia, Anastasia}. Seven. Seven male saints, seven female saints. Everybody. All of them. So in the reform after the Second Vatican Council, when they put those saints into parentheses \textemdash\ you could just say the first few and then forget the rest \textemdash\ because it seemed too Roman. The objection was, well, we want to emphasize the universal nature of holiness. It's not just Roman saints, there are all these other ones. Well, precisely! That's because the number is twelve and seven. And if we knew what the numbers meant, we wouldn't have to worry about it being Roman or universal, because it's intended to be universal, by the very nature of the number.

So, when St Benedict talks about seven canonical Hours during the day, he means praying without ceasing. And when he talks about twelve psalms at Vigils, he means praying without ceasing. So the Divine Office is rooted in the earlier monastic tradition of unceasing prayer.

There's another way that St Benedict, in an innovative way, makes this connection. Of course he relies on the tradition before him, and you know that St John Cassian in Conference 10 gives as a solution to praying without ceasing, this One-Word Prayer from the psalm, \emph{Deus in adjutorium meum intende, Domine ad adjuvandum me festina}, and I'll talk more about that later. Now St Benedict does a very interesting thing. This is also in the liturgical code, the dry and dusty part, chapter 17, verse 3. He's talking about the Day Hours: Prime, Terce, Sext, None, and so on. ``Three psalms are to be said at Prime, each followed by \emph{Glory be to the Father}. The Hymn for this Hour is sung after the opening versicle, \emph{O God, come to my assistance}, before the psalmody begins.'' Then chapter 18, verse 1: ``Each of the Day Hours begins with the verse, \emph{O God, come to my assistance, O Lord make haste to help me}.'' We're so accustomed to that, we do it all the time, we don't realize that it's an innovation here. How do we know that? You know it by your own experience, at least the older members. If you remember, I don't know what you do for the Triduum, but the Monastic Office for Holy Week is a bit curious because up until Wednesday every thing goes on just fine. Starting with Thursday it reverts to the Roman Office. The ancient Roman Office, and you begin everything \emph{absolute}, the rubric says. That is, without any introduction. So you go to choir for Terce, and it begins right away with a psalm. There's no Hymn. No \emph{O God, come to my assistance}. No nothing. \emph{Absolute}. There's a liturgical law, which the famous scholar Baumstark articulated, that the most sacred moments of the liturgical year hearken back to archaic forms. So using that archaic form of the Roman Office for those Three Days takes us back centuries and centuries. And that's how it was before St Benedict, on his initiative as far as we can tell, inserted at the beginning of the Office this psalm verse, \emph{O God, come to my assistance}. Why? Because of Cassian's teaching on prayer without ceasing. So the very fact that we begin the Office with that psalm verse connects us with the Desert Tradition. And that means that the purpose of the Office is not just to satisfy an onerous duty which the Church imposes upon us, but to enable us to pray without ceasing. And that's really exciting.

Let me switch gears now, in the 15 minutes that remain, to try and understand the Divine Office, as to its inner meaning. We've talked about the exterior expression of that in terms of number, and the \emph{O Go, come to my assistance}. When we began this exploration of prayer, we began with the Lord teaching his disciples and the affirmation that Christian prayer is really the dialogue between the Father and the Son, into which we are inserted. That is, the focus is not so much, as we ordinarily think, on us and expressing our own needs and our own emotions and our own hopes and fears and so on. But the focus is on the Trinity, and especially on the person of Christ, because the flesh is the hinge of salvation, as Tertullian says. That's how we get to the Father.

The Office has often been considered theologically as an expression of the Passion of Christ. And Augustine certainly treats it that way, but so also does Cassian. And I'd like to limit myself to him for the moment. This is from \emph{Institutes}, Book 3. He's giving a little theology of the Day Hours, and he begins with Terce: ``At the Third Hour, the Holy Spirit, which the prophets had foretold, is first seen descending upon the Apostles as they meet for divine worship.'' In the reformed Office, the Roman Office, this theology of the Offices is present in the hymns for Terce, Sext and None. It's not in the Monastic Office, but it is in the new Roman Office. So the Third Hour has this connection with the Descent of the Holy Spirit. Okay. What about the Sixth Hour? ``At the Sixth Hour, our Lord and Saviour was offered to the Father as a perfect victim and ascended the Cross for the salvation of the world.'' So the Passion of Christ is emphasized especially. Twelve o'clock, the Sixth Hour, is the time when he was crucified. ``At the same hour, the Sixth Hour, Peter was caught up in prayer, and there was revealed to him the calling of the Nations.'' This is that famous thing where the sheet is let down and there are all these animals in it. I always found that curious: ``Arise, Apostle, slay and eat.'' ``I can't do that! I don't eat those unclean things!'' ``Ah, don't call unclean what I made!'' You know the story. Great story, symbolizing the calling of the Gentiles. So at the Sixth Hour we keep in mind both Christ being crucified and the purpose for which he being was crucified, that is, the salvation of all peoples. The he talks about the Ninth Hour: ``At the Ninth Hour, the Lord burst into Hell,'' that is, the underworld. The Harrowing of Hell, going down into Hades to take out all the just, starting with Adam and Eve. In Byzantine iconography, you've probably seen those icons called \emph{The Harrowing of Hell}. Christ, radiant in white, stomping down the doors of Hades \textemdash\ t's all dark under there of course \textemdash\ and with one hand holding Adam and the other hand holding Eve. And there are the Patriarchs and so on behind them. And leading them out, leading captivity captive, as Ephesians says, and taking them up to Paradise. So Cassian meditates on the Ninth Hour in these terms, ``At the Ninth Hour, the Lord burst into Hell, and in the splendour of his countenance, dispersed the deep infernal darkness, shattered the bronze doors and broke the bars of iron.'' Cassian's a real link between East and West, as we see many times. ``He took captive the captivity of the Saints who were imprisoned in the fearful darkness of Hell.'' Of course Hell, in this sense, doesn't mean a place of punishment, it means the shadowy underworld, or Hades. ``He withdrew the firy sword,'' the angel guarding Paradise, ``and restored to Paradise its first inhabitant,'' Adam, ``who acknowledged him with love.'' Okay. But there are other references in the Scriptures to the Ninth Hour.








